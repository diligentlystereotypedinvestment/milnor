% root = book.tex
\usepackage{mathrsfs,comment}
\usepackage[usenames,dvipsnames]{color}
% \usepackage[normalem]{ulem}
\usepackage{url}
\usepackage[utf8]{inputenc}
\usepackage[T1]{fontenc}
\usepackage{textcomp}
\usepackage{amsmath, amssymb,amsthm, mathtools}

\usepackage{bm}
% \usepackage{mathpazo}
% \usepackage{euler}

% \usepackage[all,arc,2cell]{xy}
% \UseAllTwocells
\usepackage{enumerate}
\usepackage{graphicx}
% \usepackage{tikz-cd}

%%% hyperref stuff is taken from AGT style file
\usepackage{hyperref}
% \hypersetup{%
%   bookmarksnumbered=true,%
%   bookmarks=true,%
%   colorlinks=true,%
%   linkcolor=blue,%
%   citecolor=blue,%
%   filecolor=blue,%
%   menucolor=blue,%
%   pagecolor=blue,%
%   urlcolor=blue,%
%   pdfnewwindow=true,%
%   pdfstartview=FitBH}

\let\fullref\autoref
\def\makeautorefname#1#2{\expandafter\def\csname#1autorefname\endcsname{#2}}
%
%  Some standard autorefnames.  If the environment name for an autoref
%  you need is not listed below, add a similar line to your TeX file:
%
%\makeautorefname{equation}{Equation}%
\def\equationautorefname~#1\null{(#1)\null}
% \makeautorefname{footnote}{footnote}%
\makeautorefname{item}{item}%
\makeautorefname{figure}{Figure}%
\makeautorefname{table}{Table}%
\makeautorefname{part}{Part}%
\makeautorefname{appendix}{Appendix}%
\makeautorefname{chapter}{Chapter}%
\makeautorefname{section}{Section}%
\makeautorefname{subsection}{Section}%
\makeautorefname{subsubsection}{Section}%
\makeautorefname{theorem}{Theorem}%
\makeautorefname{thm}{Theorem}%
\makeautorefname{cor}{Corollary}%
\makeautorefname{lem}{Lemma}%
\makeautorefname{prop}{Proposition}%
\makeautorefname{pro}{Property}
\makeautorefname{conj}{Conjecture}%
\makeautorefname{defn}{Definition}%
% \makeautorefname{notn}{Notation}
% \makeautorefname{notns}{Notations}
\makeautorefname{rem}{Remark}%
\makeautorefname{quest}{Question}%
\makeautorefname{exmp}{Example}%
% \makeautorefname{ax}{Axiom}%
\makeautorefname{claim}{Claim}%
% \makeautorefname{ass}{Assumption}%
% \makeautorefname{asss}{Assumptions}%
% \makeautorefname{con}{Construction}%
% \makeautorefname{prob}{Problem}%
% \makeautorefname{warn}{Warning}%
\makeautorefname{obs}{Observation}%
% \makeautorefname{conv}{Convention}%
%
%                  *** End of hyperref stuff ***

% \def\th@spplain{%
%   \let\thm@indent\relax
%   \thm@headfont{\bfseries}% heading font bold face
%   \let\thmhead\thmhead@plain \let\swappedhead\swappedhead@plain
%   \thm@preskip.5\baselineskip\@plus.2\baselineskip
%                                     \@minus.2\baselineskip
%   \thm@postskip\thm@preskip
%   \normalfont
% }
% \theoremstyle{spplain}

% \newcounter{Theorem}
% \numberwithin{Theorem}{chapter}

\newtheorem{thm}{Theorem}[chapter]
\newtheorem{cor}{Corollary}[chapter]
\newtheorem{prop}{Proposition}[chapter]
\newtheorem{lem}{Lemma}[chapter]
\newtheorem{prob}{Problem}[chapter]
\newtheorem{conj}{Conjecture}[chapter]
%\newtheorem{ass}{Assumption}[section]
%\newtheorem{asses}{Assumptions}[section]
\renewcommand{\thethm}{\thechapter.\arabic{thm}}
\newcounter{def}      

\theoremstyle{definition}
\newtheorem{defn}[def]{Definition}
% \newtheorem{ass}{Assumption}[section]
% \newtheorem{asss}{Assumptions}[section]
% \newtheorem{ax}{Axiom}[section]
% \newtheorem{con}{Construction}[section]
\newtheorem{exmp}{Example}[chapter]
% \newtheorem{notn}{Notation}[section]
% \newtheorem{notns}{Notations}[section]
\newtheorem{pro}{Property}[chapter]
\newtheorem{quest}{Question}[chapter]
% \newtheorem{exercise}{Exercise}[subsection]
\newtheorem{rem*}{Remark}
\newtheorem{rem}{Remark}[chapter]
% \newtheorem{warn}{Warning}[section]
% \newtheorem{sch}{Scholium}[section]
\newtheorem{obs}[def]{Observation}
% \newtheorem{conv}{Convention}[section]
 
%%%% hack to get fullref working correctly
\makeatletter
\let\c@obs=\c@thm
\let\c@cor=\c@thm
\let\c@prop=\c@thm
\let\c@lem=\c@thm
\let\c@prob=\c@thm
\let\c@con=\c@thm
\let\c@conj=\c@thm
\let\c@defn=\c@thm
\let\c@notn=\c@thm
\let\c@notns=\c@thm
\let\c@exmp=\c@thm
\let\c@ax=\c@thm
\let\c@pro=\c@thm
\let\c@ass=\c@thm
\let\c@warn=\c@thm
\let\c@rem=\c@thm
\let\c@sch=\c@thm
\let\c@equation\c@thm
\numberwithin{equation}{section}
\makeatother

\newcommand{\Z}{\bm{Z}}
\newcommand{\A}{\bm{A}}
\newcommand{\F}{\bm{F}}
\newcommand{\Q}{\bm{Q}}
\newcommand{\R}{\bm{R}}
\newcommand{\C}{\bm{C}}
\newcommand{\N}{\bm{N}}
\newcommand{\SL}{\text{SL}}
\newcommand{\GL}{\text{GL}}
\newcommand{\Id}{\text{Id}}
\newcommand{\Gr}{\text{Gr}}
\DeclareMathOperator{\grad}{grad}
\DeclareMathOperator{\tr}{tr}
\DeclareMathOperator{\Hom}{Hom}
\DeclareMathOperator{\Gal}{Gal}
\DeclareMathOperator{\Lie}{Lie}
\DeclareMathOperator{\PGL}{PGL}
\DeclareMathOperator{\Wedge}{\bigwedge}
\DeclareMathOperator{\Res}{Res}
\DeclareMathOperator{\sgn}{sgn}
\renewcommand{\S}{\mathbb{S}}
\renewcommand{\char}{\text{char}}
\DeclareMathOperator{\im}{im}
\renewcommand{\P}{\mathbb{P}}
\DeclareMathOperator*{\colim}{colim}
\DeclareMathOperator*{\rank}{rank}
\DeclareMathOperator{\Aut}{Aut}
\DeclareMathOperator{\Mor}{Mor}
\DeclareMathOperator{\Spec}{Spec}
\DeclareMathOperator{\Supp}{Supp}
\newcommand{\CPn}{\C P^n}
\newcommand{\CP}{\C P}
\newcommand{\RPn}{\R P^n}
\newcommand{\RP}{\R P}
\newcommand{\MyTo}[1]{\mathbin{\,\tikz[baseline] \draw[-stealth,line width=.4pt] (0ex,0.4ex) -- (#1,0.4ex);}}
\newcommand{\Dlim}{%
    \mathchoice
      {\lim_{\MyTo{3.0ex}}}% \displaystyle
      {\lim_{\MyTo{2.5ex}}}% \textstyle
      {\lim_{\MyTo{2.0ex}}}% \scriptstyle
      {\lim_{\MyTo{2.0ex}}}% \scriptscriptstyle
}
\DeclareMathOperator{\Tor}{Tor}
\DeclareMathOperator{\height}{height}
\DeclareMathOperator{\Ann}{Ann}
\DeclareMathOperator{\End}{End}
\DeclareMathOperator{\coker}{coker}
\DeclareMathOperator{\Max}{Max}
\DeclareMathOperator{\Pic}{Pic}
