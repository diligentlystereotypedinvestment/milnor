% root = book.tex
\chapter{Projective Modules and $K_{0} \Lambda $}
\label{chap1}

The word ring will always mean associative ring with an identity element $1$.

Consider left modules over a ring $\Lambda$. Recall that a module $M$ is \textbf{free} if there exists a basis $\{m_{\alpha}\}$ so that each module element can be expressed uniquely as a finite sum $\sum \lambda_{\alpha} m_{\alpha}$, and \textbf{projective} if there exists a module $N$ so that the direct sum $M \oplus N$ is free. This is equivalent to the requirement that every short exact sequence
\[
	0 \to X \to Y \to M \to 0
\]
must be split exact, so that $Y \cong X \oplus M$.

The \textbf{projective module group} $K_{0}\Lambda$ is an additive group defined by generators and relations as follows. There is to be one generator $[P]$ for each isomorphism class of finitely generated projective modules $P$ over $\Lambda$, and one relation
\[
[P]+[Q]=[P \oplus Q]
\]

for each pair of finitely generated projectives. (Compare the proof of \autoref{lem:1.1} below.)

Clearly every element of $K_{0} \Lambda$ can be expressed as the difference $\left[P_{1}\right]-\left[P_{2}\right]$ of two generators. (In fact, adding the same projective module to $P_{1}$ and $P_{2}$ if necessary, we may even assume that $P_{2}$ is free.) We will give a criterion for the equality of two such differences.

First another definition. Let $\Lambda^{r}$ denote the free module consisting of all $r$-tuples of elements of $\Lambda$. Two modules $M$ and $N$ are called \textbf{stably isomorphic} if there exists an integer $r$ so that
\[
M \oplus \Lambda^{r} \cong N \oplus \Lambda^{r}
\]

\begin{lem}\label{lem:1.1}
	The generator $[P]$ of $K_{0} \Lambda$ is equal to the generator $[Q]$ if and only if $P$ is stably isomorphic to $Q$ . Hence the difference $\left[P_{1}\right]-\left[P_{2}\right]$ is equal to $\left[Q_{1}\right]-\left[Q_{2}\right]$ if and only if $P_{1} \oplus Q_{2}$ is stably isomorphic to $P_{2} \oplus Q_{1}$.
\end{lem}

\begin{proof}
	The group $K_{0} \Lambda$ can be defined more formally as a quotient group $F / R$, where $F$ is free abelian with one generator $\langle P\rangle$ for each isomorphism class of finitely generated projectives $P$, and where $R$ is the subgroup spanned by all $\langle P\rangle+\langle Q\rangle-\langle P \oplus Q\rangle$. (Thus we are reserving the symbol $[P]$ for the residue class of $\langle P\rangle$ modulo $R$.) Note that a sum $\left\langle P_{1}\right\rangle+\cdots+\left\langle P_{k}\right\rangle$ in $F$ is equal to $\left\langle Q_{1}\right\rangle+\cdots+\left\langle Q_{k}\right\rangle$ if and only if
\[
P_{1} \cong Q_{\pi(1)}, \ldots, P_{k} \cong Q_{\pi(k)}
\]
for some permutation $\pi$ of $\{1, \ldots, k\}$. If this is the case, note the resulting isomorphism
\[
P_{1} \oplus \cdots \oplus P_{k} \cong Q_{1} \oplus \cdots \oplus Q_{k}
\]

Now suppose that $\langle M\rangle \equiv\langle N\rangle \pmod{R}$. This means that

\begin{align*}
\langle M>-\langle N\rangle &=\sum\left(\left\langle P_{i}\right\rangle+\left\langle Q_{\mathfrak{i}}\right\rangle-\left\langle P_{i} \oplus Q_{\dot{i}}\right\rangle\right) \\
& -\sum\left(\left\langle P_{\mathfrak{j}}'\right\rangle+\left\langle Q_{j}'\right\rangle-\left\langle P_{j}' \oplus Q_{j'}'\right\rangle\right)
\end{align*}

for appropriate modules $P_{i}, Q_{i}, P_{j}', Q_{j}'$.

Transposing all negative terms to the opposite side of the equation and then applying the remark above, we get
\[
	M \oplus\left(\sum\left(P_{i} \oplus Q_{i}\right) \oplus \sum P_{j}' \oplus \sum Q_{j}'\right) \cong N \oplus\left(\sum P_{i} \oplus \sum Q_{i} \oplus \sum\left(P_{j}' \oplus Q_{j}'\right)\right),
\]
or briefly $M \oplus X \cong N \oplus X$, since the expressions inside the long parentheses are clearly isomorphic. Now choose $Y$ so that $X \oplus Y$ is free, say $X \oplus Y \cong \Lambda^{r}$. Then adding $Y$ to both sides we obtain $M \oplus \Lambda^{r} \cong N \oplus \Lambda^{r}$. Thus M is stably isomorphic to N .

The rest of the proof of \autoref{lem:1.1} is straightforward.
\end{proof}

If the ring $\Lambda$ is commutative, note that the tensor product over $\Lambda$ of (finitely generated projective) left $\Lambda$-modules is again a (finitely generated projective) left $\Lambda$ module. Defining
\[
[P] \cdot[Q]=[P \otimes Q]
\]
we make the additive group $K_{0} \Lambda$ into a commutative ring. The identity element of this ring is the class $\left[\Lambda^{1}\right]$ of the free module on one generator.

In order to compute the group $K_{0} \Lambda$ it is necessary to ask two questions.

\begin{quest}\label{quest:1.1.1}
	Is every finitely generated projective over $\Lambda$ actually free (or at least stably free)?
\end{quest}

\begin{quest}\label{quest:1.1.2}
	Is the number of elements in a basis for a free module actually an invariant of the module? In other words if $\Lambda^{r} \cong \Lambda^{S}$ does it follow that $r=s$?
\end{quest}

\textit{If both questions have an affirmative answer then clearly $K_{0} \Lambda$ is the free abelian group generated by $\left[\Lambda^{1}\right]$. This will be true, for example, if $\Lambda$ is a field, or a skew field, or a principal ideal domain.}

Of course \fullref{quest:1.1.1} and \fullref{quest:1.1.2} may have negative answers. For example if $\Lambda$ is the ring of endomorphisms of a finite dimensional vector space of dimension greater than $1$ , then \Cref{quest:1.1.1} has a negative answer; and if $\Lambda$ is the ring of endomorphisms of an infinite dimensional vector space then \Cref{quest:1.1.2} has a negative answer. (The group $K_{0} \Lambda$ is infinite cyclic but not generated by $[\Lambda^{1}]$ in the first case, and is zero in the second.)

Here is an important example in which $K_{0} \Lambda$ is free cyclic.

\begin{lem}\label{lem:1.2}
	If $\Lambda$ is a local ring, then every finitely generated\footnote{Compare \cite{kaplansky:projective}} projective is free, and $K_{0} \Lambda$ is the free cyclic group generated by $\left[\Lambda^{1}\right]$.
\end{lem}

\begin{proof}
	First recall the relevant definitions. A ring element $u$ is called a unit if there exists a ring element $v$ with $u v=v u=1$. The set $\Lambda^{\bullet}$ consisting of all units in $\Lambda$ evidently forms a multiplicative group.

	$\Lambda$ is called a \textbf{local ring} if the set $\mathfrak{m}=\Lambda-\Lambda^{\bullet}$ consisting of all nonunits is a left ideal. It follows that $\mathfrak{m}$ is a right ideal also. For if some product $m \lambda$ with $m \in \mathfrak{m}$ and $\lambda \in \Lambda$ were a unit, then clearly $m$ would have a right inverse, say $m v=1$. This element $v$ certainly cannot belong to the left ideal $\mathfrak{m}$ . But $v$ cannot be a unit either. For if $v$ were a unit, then the computation
	\[
	m=m\left(v^{-1}\right)=(mv) v^{-1}=v^{-1}
	\]
	would show that $m$ must be a unit.

	This contradiction shows that $\mathfrak{m}$ is indeed a two-sided ideal. The quotient ring $\Lambda / \mathfrak{m}$ is evidently a field or skew-field.

	Note that a square matrix with entries in $\Lambda$ is non-singular if and only if the corresponding matrix with entries in the quotient $\Lambda / \mathfrak{m}$ is non-singular. To prove this fact, multiply the given matrix on the left by a matrix which represents an inverse modulo $\mathfrak{m}$, and then apply elementary row operations to diagonalize. This shows that the matrix has a left inverse, and a similar argument constructs a right inverse.

	We are now ready to prove \Cref{lem:1.2}. If the module $P$ is finitely generated and projective over $\Lambda$ then we can choose $Q$ so that $P \oplus Q \cong \Lambda^{r}$. Thinking of the quotients $P / \mathfrak{m}P$ and $Q / \mathfrak{m}Q$ as vector spaces over the skew-field $\Lambda / \mathfrak{m}$, we can choose bases. Choose a representative in $P$ or in $Q$ for each basis element. The above remark on matrices then implies that the elements so obtained constitute a basis for $P \oplus Q$. Clearly it follows that $P$ and $Q$ are free. Since the dimension of the vector space $P / \mathfrak{m}P$ is an invariant of $P$ , this completes the proof.
\end{proof}

Next consider a homomorphism
\[
f: \Lambda \to \Lambda'
\]
between two rings. (It is always assumed that $f(1)=1$.) Then every module $M$ over $\Lambda$ gives rise to a module
\[
f_{\#} M=\Lambda' \otimes \Lambda^{M}
\]

over $\Lambda'$. Clearly if $M$ is finitely generated, or free, or projective, or splits as a direct sum over $\Lambda$, then $f_{\#} M$ is finitely generated, or free, or projective, or splits as as a corresponding direct sum over $\Lambda'$. Hence the correspondence
\[
[P] \mapsto\left[f_{\#} P\right]
\]
gives rise to a homomorphism
\[
f_{ast}: K_{0} \Lambda \to K_{0} \Lambda'
\]
of abelian groups. Note the functorial properties
\[
(\text{identity})_{ast}=\text{identity} \qquad (f \circ g)_{ast}=f_{ast} \circ g_{ast}.
\]

\begin{exmp}\label{ex:1.1}
	Let $\Z$ be the ring of integers. Then for any ring $\Lambda$ there is a unique homomorphism
	\[
		i: Z \to \Lambda
	\]
	The image
	\[
		i_{ast} \mathrm{~K}_{0} Z \subset K_{0} \Lambda
	\]
	is clearly the subgroup generated by the free module [ $\Lambda^{1}$ ]. The co-kernel

	\[
	K_{0} \Lambda /\left(\text { subgroup generated by }\left[\Lambda^{1}\right]\right)=K_{0} \Lambda / i_{ast} \mathrm{~K}_{0} Z
	\]

	is called the projective class group of $\Lambda$.
\end{exmp}

\begin{exmp}\label{ex:1.2}
	Suppose that $\Lambda$ can be mapped homomorphically into a field or skew-field $F$. This is always possible, for example, if $\Lambda$ is commutative. Then we obtain a homomorphism
	\[
		j_{ast}: K_{0} \Lambda \to K_{0} \mathrm{~F} \cong \Z
	\]
	In the commutative case, this homomorphism is clearly determined by the kernel of $j$, which is a prime ideal in $\Lambda$. Hence one can speak of the \textbf{rank} of a projective module at a prime ideal $\mathfrak{p}$. If $\mathfrak{p} \supset \mathfrak{p}'$, note that the rank at $p$ is equal to the rank at $p'$. For if we localize the integral domain $\Lambda / p'$ at the ideal corresponding to $p$ (that is adjoin the inverses of all elements not belonging to $\mathfrak{p}$) we obtain a local ring which embeds in the quotient field of $\Lambda / p'$ and maps homomorphically into the quotient field of $\Lambda / \mathfrak{p}$. Using Lemma 1.2, it follows that the ranks are equal. \textit{In particular, if $\Lambda$ is an integral domain, then the rank of a projective module is the same at all prime ideals.}

	In any case, choosing some fixed homomorphism $j: \Lambda \to F$, since $j_{ast} i_{ast}$ is an isomorphism, we obtain a direct sum decomposition
	\[
		K_{0} \Lambda= \im i_\ast \oplus \ker j_\ast
	\]
	The first summand is free cyclic, and the second maps bijectively to the projective class group of $\Lambda$.

	In the commutative case, note that $\ker j_\ast $ is an ideal in the ring $K_{0} \Lambda$. We will denote this ideal by $\tilde{K}_{0} \Lambda$, and write
	\[
		K_{0} \Lambda \cong \Z \oplus \tilde{K}_{0} \Lambda.
	\]
\end{exmp}

\begin{exmp}
	Suppose that $\Lambda$ splits as a cartesian product
	\[
		\Lambda_{1} \times \Lambda_{2} \times \cdots \times \Lambda_{k}
	\]
	of rings. Then the projection homomorphisms
	\[
		K_{0} \Lambda \to K_{0} \Lambda_{i}
	\]
	give rise to a corresponding cartesian product structure

	\[
		K_{0} \Lambda \cong K_{0} \Lambda_{1} \times K_{0} \Lambda_{2} \times \ldots \times K_{0} \Lambda_{k}
	\]
	The proof is not difficult.

	Such a splitting of $\Lambda$ occurs for example whenever $\Lambda$ is commutative and artinian,\footnote{A ring is \textbf{artinian} if every descrending sequence of ideals must terminate.} but is not local. For since $\Lambda$ is commutative, the set of all nilpotent elements forms an ideal. If $\Lambda$ is not local, there must exist an element $\lambda$ which is neither a unit nor a nilpotent element. Since $\Lambda$ is artinian, the sequence of principal ideals
	\[
		(\lambda) \supset\left(\lambda^{2}\right) \supset\left(\lambda^{3}\right) \supset \cdots
	\]
	must terminate, say $\left(\lambda^{n}\right)=\left(\lambda^{n+1}\right)=\cdots$ so that $\lambda^{n}=\rho \lambda^{2 n}$ for some $\rho$. But this implies that the element $e=\rho \lambda^{n}$ is idempotent ( $ee=e$ ), and hence that $\Lambda$ splits as a cartesian product
	\[
		\Lambda \cong \Lambda /(e) \times \Lambda /(1-e)
	\]
	This splitting is not trivial since the hypothesis that $\lambda$ is neither a unit nor nilpotent implies that $e \neq 1,0$. This procedure can be continued inductively until $\Lambda$ has been expressed as a cartesian product of local rings. It then follows that
	\[
	K_{0} \Lambda \cong \Z \times \Z \times \ldots \times \Z
	\]
\end{exmp}

\section{Dedekind Domains}

Important examples in which the ring $K_{0} \Lambda$ has a more interesting structure are provided by Dedekind domains. We will discuss these in some detail, starting for variety with a non-standard version of the definition.
\footnote{The usual definition is of course equivalent to the one given here. For further information, see \cite{Zariski}; or \cite{Lang}; as well as \cite{Cartan1999Dec}}

\begin{defn}
	A \textbf{Dedekind domain} is a commutative ring without zero divisors such that, for any pair of ideals $\mathfrak{a} \subset \mathfrak{b}$, there exists an ideal $\mathfrak{c}$ with $\mathfrak{a}=\mathfrak{b} \mathfrak{c}$.
\end{defn}

\begin{rem}\label{rem:1.3}
	Note that the ideal $\mathfrak{c}$ is uniquely determined, except in the trivial case $\mathfrak{a}=\mathfrak{b}=0$. In fact if $\mathfrak{b} \mathfrak{c}=\mathfrak{b} \mathfrak{c}'$, then choosing some nonzero principal ideal $b_{0} \Lambda \subset \mathfrak{b}$ we can express $b_{0} \Lambda$ as a product $\mathfrak{x} \mathfrak{b}$ and conclude that $\mathfrak{x} \mathfrak{b} \mathfrak{c}=\mathfrak{x} \mathfrak{b} \mathfrak{c}'$, hence $\mathfrak{b}_{0} \mathfrak{c}=\mathfrak{b}_{0} \mathfrak{c}'$, from which the equality $\mathfrak{c}=\mathfrak{c}'$ follows.
\end{rem}

\begin{defn}
	Two non-zero ideals $\mathfrak{a}$ and $\mathfrak{b}$ in the Dedekind domain $\Lambda$ belong to the same \textbf{ideal class} if there exist non-zero ring elements x and y so that $xa=yb$.
\end{defn}

Clearly the ideal classes of $\Lambda$ form an abelian group under multiplication, with the class of principal ideals as identity element. We will use the notation $C(\Lambda)$ for the ideal class group of $\Lambda$, and the notation $\{\mathfrak{a}\} \in C(\Lambda)$ for the ideal class of $\mathfrak{a}$.

Note that $\{\mathfrak{a}\}=\{\mathfrak{b}\}$ if and only if $\mathfrak{a}$ is isomorphic, as $\Lambda$-module, to $\mathfrak{b}$. For if $\phi: \mathfrak{a} \to \mathfrak{b}$ is an isomorphism, then choosing $a_{0} \in \mathfrak{a}$, the computation $a_{0} \phi(a)=\phi\left(a_{0} a\right)=\phi\left(a_{0}\right) a$ shows that $a_{0} \mathfrak{b}=\phi\left(a_{0}\right) \mathfrak{a}$.

Important examples of Dedekind domains can be constructed as follows. Let $F$ be a finite extension of the field $\Q$ of rational numbers. An element of $F$ is called an \textbf{algebraic integer} if it is the root of a monic polynomial
\[
x^{k}+a_{1} x^{k-1}+\cdots+a_{k}
\]
with coefficients $a_{i} \in \Z$.

\begin{thm}\label{thm:1.4}
	The set $\Lambda=\Lambda(F)$ consisting of all algebraic integers in F is a Dedekind domain, with quotient field $F$.
\end{thm}

The proof of this classical theorem will be deferred until the end of $\Cref{chap1}$.

For such a ring $\Lambda(F)$ of algebraic integers, the ideal class group $C(\Lambda(F)$ ) is always finite. (See for example \cite{hecke:1970}; or \cite{Hunter_1969}) As examples, for the domains $\Z[i], \Z[\sqrt{-5}]$, and $\Z\left[ \frac{1+\sqrt{-23}}{2} \right]$, the ideal class group has order $1,2$, and $3$ respectively. Further examples will be given in Chapter 3.4.

Projective modules over Dedekind domains can be classified as follows.

\begin{lem}\label{lem:1.5}
	Every ideal in a Dedekind domain $\Lambda$ is finitely generated and projective over $\Lambda$. Conversely every finitely generated projective module over $\Lambda$ is isomorphic to a direct sum $\mathfrak{a}_{1} \oplus \cdots \oplus \mathfrak{a}_{k}$ of ideals.
\end{lem}

\begin{proof}
	If $\mathfrak{b}$ is a non-zero ideal, choose $0 \neq a_{0} \in \mathfrak{b}$, so $a_{0} \Lambda \subset \mathfrak{b}$, and define $\mathfrak{c}$ by the equality $a_{0} \Lambda=\mathfrak{b} c$. Then the generator $a_{0}$ can be expressed as a finite sum $b_{1} c_{1}+\cdots+b_{k} c_{k}$ with $b_{i} \in \mathfrak{b}, c_{i} \in \mathfrak{c}$. Define $\Lambda$-linear mappings
\[
\mathfrak{b} \to \Lambda^{k} \text{ and } \Lambda^{k} \to \mathfrak{b}
\]
by the formulas $b \mapsto\left(bc_{1} / a_{0}, \ldots, bc_{k} / a_{0}\right)$ and $\left(x_{1}, \ldots, x_{k}\right) \mapsto b_{1} x_{1}+\ldots+b_{k} x_{k}$. Since the composition is the identity map of $\mathfrak{b}$, this proves that $\mathfrak{b}$ is finitely generated and projective.

Any finitely generated projective $P$ can be embedded in the free module $\Lambda^{k}$ for some $k$. Projecting to the $k$-th factor we obtain a homomorphism $\phi: P \to \Lambda$ with $\ker\phi \subset \Lambda^{k-1}$.

Since the image $\phi(P)=\mathfrak{a}_{k}$ is an ideal, hence projective, we have $P \cong \ker \phi \oplus \mathfrak{a}_{k}$. An easy induction now completes the proof.
\end{proof}

\begin{rem*}
	More generally, any module which is finitely generated and torsion free over $\Lambda$ can easily be embedded in some $\Lambda^{k}$ and hence, by this argument, is projective.
\end{rem*}

\begin{thm}
	[Steiniz]
 \label{thm:1.6}
	Two direct sums $\mathfrak{a}_{1} \oplus \cdots \oplus \mathfrak{a}_{r}$ and $\mathfrak{b}_{1} \oplus \cdots \oplus \mathfrak{b}_{s}$ of non-zero ideals are isomorphic as $\Lambda$-modules if and only if $r=s$ and the ideal class $\left\{\mathfrak{a}_{1} \mathfrak{a}_{2} \cdots \mathfrak{a}_{r}\right\}$ is equal to $\left\{\mathfrak{b}_{1} \mathfrak{b}_{2} \cdots \mathfrak{b}_{r}\right\}$
\end{thm}

(Compare \cite{kaplansky:1952})

\begin{proof}
For the first half of the proof, the ring $\Lambda$ can be any integral domain. First note that, if $\mathfrak{a} \subset \Lambda$ is a non-zero ideal, then any $\Lambda$-linear mapping $\phi: \mathfrak{a} \to \mathfrak{b} \subset \Lambda$ determines a unique element $q$ of the quotient field of $\Lambda$ such that
\[
	\phi(a)=q a \forall a \in \mathfrak{a}
\]
To prove this it is only necessary to divide the equation $a_{0} \phi(a)=\phi\left(a_{0} a\right)=$ $\phi\left(a_{0}\right) a$ by $a_{0}$, setting $q=\phi\left(a_{0}\right) / a_{0}$. Similarly, if the mapping
\[
\phi: \mathfrak{a}_{1} \oplus \cdots \oplus \mathfrak{a}_{r} \to \mathfrak{b}_{1} \oplus \cdots \oplus \mathfrak{b}_{s}
\]
is $\Lambda$-linear, then there is a unique $s \times r$ matrix $Q=\left(q_{ij}\right)$ with entries in the quotient field so that the $i$-th component of $\phi\left(a_{1}, \ldots, a_{r}\right)=\left(b_{1}, \ldots, b_{S}\right)$ is
\[
b_{i}=\sum q_{i j} a_{j}
\]
for all $\left(a_{1}, \ldots, a_{r}\right) \in \mathfrak{a}_{1} \oplus \cdots \oplus \mathfrak{a}_{r}$. If $\phi$ is an isomorphism, then this matrix $Q$ has an inverse, hence $r=s$. We then assert that the product ideal $\mathfrak{b}_{1} \cdots \mathfrak{b}_{r}$ is equal to $\det(Q) \mathfrak{a}_{1} \cdots \mathfrak{a}_{r}$. In fact for each generator $a_{1} \ldots a_{r}$ of $a_{1} \ldots a_{r}$, the product $\det(Q) a_{1} \ldots a_{r}$ can be expressed as the determinant of the product matrix
\[
Q
\left(\begin{bmatrix}
a_{1} & 0 & \ldots & 0 \\
0 & a_{2} & \cdots & 0 \\
\vdots & \vdots & \ddots & \vdots \\
\vdots & \vdots & & \vdots \\
0 & 0 & \ldots & a_{r}
\end{bmatrix}\right).
\]
whose $i$-th row consists completely of elements $q_{i j} a_{j}$ of $\mathfrak{b}_{i}$. This proves that
\[
	\det(Q) \mathfrak{a}_{1} \cdots \mathfrak{a}_{r} \subset \mathfrak{b}_{1} \cdots \mathfrak{b}_{i}
\]
A similar argument shows that
\[
\det(Q^{-1} \mathfrak{b}_{1} \cdots \mathfrak{b}_{r} \subset \mathfrak{a}_{1} \cdots \mathfrak{a}_{r}
\]
Multiplying this last inclusion by $\det(Q)$ and comparing, it follows that $\mathfrak{b}_{1} \cdots \mathfrak{b}_{r}$ is equal to $\det(Q) \mathfrak{a}_{1} \cdots \mathfrak{a}_{r}$; and hence belongs to the ideal class $\left\{a_{1} \cdots a_{r}\right\}$. This proves the first half of \autoref{thm:1.6}.

To prove that the rank $r$ and the ideal class $\left\{\mathfrak{a}_{1} \cdots \mathfrak{a}_{r}\right\}$ form a complete invariant for $\mathfrak{a}_{1} \oplus \cdots \oplus \mathfrak{a}_{r}$, it clearly suffices to prove the following.

\begin{lem}\label{1.7}
	If $\mathfrak{a}$ and $\mathfrak{b}$ are non-zero ideals in a Dedekind domain $\Lambda$, then the module $\mathfrak{a} \oplus \mathfrak{b}$ is isomorphic to $\Lambda^{1} \oplus(\mathfrak{a} \mathfrak{b})$.
\end{lem}

\begin{proof}
	If $\mathfrak{a}$ and $\mathfrak{b}$ happen to be relatively prime $(\mathfrak{a}+\mathfrak{b}=\Lambda)$, the proof proceeds as follows. Map $\mathfrak{a} \oplus \mathfrak{b}$ onto $\Lambda^{1}$ by the correspondence $\mathfrak{a} \oplus \mathfrak{b} \mapsto \mathfrak{a}+\mathfrak{b}$. The kernel is clearly isomorphic to the module $\mathfrak{a} \cap \mathfrak{b}$. Since $\Lambda^{1}$ is projective, the sequence $0 \to \mathfrak{a} \cap \mathfrak{b} \to \mathfrak{a} \oplus \mathfrak{b} \to \Lambda^{1} \to 0$ is split exact, and therefore $\mathfrak{a} \oplus \mathfrak{b} \cong \Lambda^{1} \oplus (\mathfrak{a} \cap \mathfrak{b})$. But the intersection $\mathfrak{a} \cap \mathfrak{b}$ is equal to the product ideal $\mathfrak{a} \mathfrak{b}$. For the inclusion $\mathfrak{a} \mathfrak{b} \subset \mathfrak{a} \cap \mathfrak{b}$ is clear; and if $1=a_{0}+b_{0}$ then every $x \in \mathfrak{a} \cap \mathfrak{b}$ can be expressed as $x=a_{0} x+x b_{0}$, and hence belongs to $\mathfrak{a} \mathfrak{b}$. Thus $\mathfrak{a} \oplus \mathfrak{b} \cong \Lambda^{1} \oplus \mathfrak{a} \mathfrak{b}$ as required.

	For the general case, the hypothesis that $\Lambda$ is a Dedekind domain will be needed in order to replace $\mathfrak{a}$ by an ideal which is relatively prime to $\mathfrak{b}$. It clearly suffices to prove the following.

\begin{lem}\label{lem:1.8}
	Given non-zero ideals $\mathfrak{a}$ and $\mathfrak{b}$ in a Dedekind domain $\Lambda$ there exists an ideal $\mathfrak{a}'$ in the ideal class of $\mathfrak{a}$ which is prime to $\mathfrak{b} $.
\end{lem}

To prove this we must first establish two of the standard properties of Dedekind domains.

\begin{lem}\label{lem:1.9}
	Every non-zero ideal in a Dedekind domain $\Lambda$ can be expressed uniquely as a product of maximal ideals.
\end{lem}

\begin{proof}
	In fact choosing any maximal ideal $\mathfrak{m}_{1} \supset \mathfrak{a}$ we have $\mathfrak{a}=\mathfrak{m}_{1} \mathfrak{a}_{1}$ for some ideal $\mathfrak{a}_{1}$, then similarly $\mathfrak{a}_{1}=\mathfrak{m}_{2} \mathfrak{a}_{2}$, and so on, with $\mathfrak{a} \subset \mathfrak{a}_{1} \subset \mathfrak{a}_{2} \subset \cdots$ This sequence must terminate, since $\Lambda$ is Noetherian by \fullref{lem:1.5}. The resulting factorization is unique. For if $\mathfrak{m}_{1} \cdots \mathfrak{m}_{k}=\mathfrak{m}_{1}' \cdots \mathfrak{m}_{\ell}'$ then $\mathfrak{m}_{1}' \supset \mathfrak{m}_{1} \cdots \mathfrak{m}_{k}$ and hence, since $\mathfrak{m}_{1}'$ is prime, $\mathfrak{m}_{1}'$ contains some $\mathfrak{m}_{i}$, and therefore is equal to $m_{i}$. The uniqueness statement then follows by induction on $\Max(k, \ell)$, using \fullref{rem:1.3}.
\end{proof}

\begin{lem}\label{lem:1.10}
	For any non-zero ideal $\mathfrak{a} $ in a Dedekind domain $\Lambda$, the quotient $\Lambda / \mathfrak{a}$ is a principal ideal ring (usually with zero divisors).
\end{lem}
\begin{proof}
	Let $\mathfrak{m}_{1}, \ldots, \mathfrak{m}_{k}$ be the distinct maximal ideals containing $\mathfrak{a}$. We will first show that each $\mathfrak{m}_{i}$ is a principal ideal modulo $\mathfrak{a}$. Let $x_{1}$ be a ring element which belongs to $\mathfrak{m}_{1}$ but not to $\mathfrak{m}_{1}^{2}$. Since the ideals $\mathfrak{m}_{1}^{2}, \mathfrak{m}_{2}, \ldots, \mathfrak{m}_{k}$ are pairwise relatively prime (using \fullref{lem:1.9}), it follows that there exists a ring element $y_{1}$ so that
	\begin{align*}
		y_{1} &\equiv x_{1} \pmod{\mathfrak{m}_{1}^{2}} \\
		y_{1} &\equiv 1 \pmod{\mathfrak{m}_{j}} \text { for } j>1
	\end{align*}
	using the Chinese Remainder Theorem. (See for example \cite{Lang2012}) Then the ideal generated by $y_{1}$ and $\mathfrak{a}$ is contained in $\mathfrak{m}_{1}$, but is not contained in $\mathfrak{m}_{1}^{2}$ or in any other maximal ideal. So, using \fullref{lem:1.9}, this ideal can only be $\mathfrak{m}_{1}$ itself.

	This proves that $\mathfrak{m}_{1}$ is a principal ideal modulo $\mathfrak{a}$. But every ideal of $\Lambda / a$ is a product of maximal ideals, so this completes the proof of \fullref{lem:1.10}.
\end{proof}

We are now ready to prove \fullref{lem:1.8}. Given non-zero ideals $\mathfrak{a}$ and $\mathfrak{b}$, choose $0 \neq a_{0} \in \mathfrak{a}$ and define $\mathfrak{x}$ by the equation $\mathfrak{x} \mathfrak{a}=a_{0} \Lambda$. Applying \fullref{lem:1.10} to the ideal $\mathfrak{x}$ modulo $\mathfrak{b} x$, we see that $\mathfrak{x}$ is generated by $\mathfrak{b} x$ together with some element $x_{0}$. Now multiplying the equation
\[
\mathfrak{x}=\mathfrak{b} \mathfrak{x}+\mathfrak{x}_{0} \Lambda
\]
by $\mathfrak{a}$, and then dividing by $a_{0}$, we obtain
\[
\Lambda=\mathfrak{b}+\mathfrak{a} x_{0} / a_{0}
\]
Since $\mathfrak{a} x_{0} / a_{0}$ is clearly an ideal in the ideal class $\{\mathfrak{a}\}$, this proves \fullref{lem:1.8}.
\end{proof}
Hence this completes the proof of \fullref{thm:1.6}.
\end{proof}

\begin{cor}\label{1.11}
	If $\Lambda$ is a Dedekind domain, then $K_{0} \Lambda \cong \Z \oplus \tilde{K}_{0} \Lambda$, where the additive group of $\tilde{K}_{0} \Lambda$ is canonically isomorphic to the ideal class group $C(\Lambda)$, and where the product of any two elements in the ideal $\tilde{K}_{0} \Lambda$ is zero.
\end{cor}

\begin{proof}
	In fact the correspondence
	\[
		\left[\mathfrak{a}_{1} \oplus \cdots \oplus \mathfrak{a}_{r}\right] \mapsto\left(r,\left\{\mathfrak{a}_{1} \cdots \mathfrak{a}_{r}\right\}\right)
	\]
	maps $K_{0} \Lambda$ isomorphically onto $\Z \oplus C(\Lambda)$. Recall that $\tilde{K}_{0} \Lambda$ can be identified with the set of differences $[P]-[Q]$ with $\rank P=\rank Q$. Then each element of $\tilde{K}_{0} \Lambda$ can be written as a difference $[\mathfrak{a}]-\left[\Lambda^{1}\right]$, and we must prove that
	\[
		\left([\mathfrak{a}]-\left[\Lambda^{1}\right]\right)\left([\mathfrak{b}]-\left[\Lambda^{1}\right]\right)=0
	\]
	But the product $[\mathfrak{a}][\mathfrak{b}]=[\mathfrak{a} \otimes \mathfrak{b}]$ is equal to $[\mathfrak{a} \mathfrak{b}]$. In fact the projective modules $\mathfrak{a} \otimes \mathfrak{b}$ and $\mathfrak{a} \mathfrak{b}$ both have rank $1$, so the natural surjection from the tensor product to the product ideal is an isomorphism. The conclusion now follows from \fullref{lem:1.7}.
\end{proof}

\begin{rem*}
	The ideal class group $C(\Lambda)$ can be naturally identified with a multiplicative group, $1+\tilde{K}_{0} \Lambda$, of units in the ring $K_{0} \Lambda$. Something similar happens for an arbitrary commutative ring. Call a module M , over a commutative ring $\Lambda$, \textbf{invertible} if there exists a module $N$ so that $M \otimes N$ is free on one generator. The set of isomorphism classes of invertible modules clearly forms a group under the tensor product. This group is called the \textbf{Picard group}, denoted by $\Pic(\Lambda)$.

	It can be shown that a module is invertible if and only if is projective, finitely generated, and has rank $1$ at every prime ideal. (Compare \cite{bourbaki143}) Furthermore the second exterior power $E_{\Lambda}^{2} M$ of an invertible module is zero. For this exterior power is a projective module which has rank zero at every prime. (\cite{bourbaki112}) It follows that the Picard group embeds as a subgroup of the group of units in $K_{0} \Lambda$. For if two invertible modules $M$ and $M'$ are stably isomorphic, $M \oplus \Lambda^{r} \cong M' \oplus \Lambda^{r}$, then taking the $(r+1)$-st exterior power of each side, we see that $M \cong M'$. (Bass proves the sharper statement that there exists a canonical retracting homomorphism from the additive group of $K_{0} \Lambda$ to the multiplicative group $\Pic(\Lambda) \subset K_{0} \Lambda$.)

	In the case of a Dedekind domain, it is clear that $\Pic(\Lambda)$ is canonically isomorphic to the ideal class group $C(\Lambda)$.
\end{rem*}

To conclude Chapter 1, let us prove \autoref{thm:1.4}. If $F$ is a finite extension of the field of rational numbers, we must show that the set $\Lambda$, consisting of all algebraic integers in $F$, is a Dedekind domain.

Let $n$ be the degree of $F$ over $\Q$ . It will be convenient to use the word lattice to mean an additive subgroup of $F$ which has a finite basis. Thus every lattice $L$ in $F$ is a free abelian additive group of rank $\leq n$. The product $LL'$ of two lattices in $F$ is the lattice generated by all products $\ell \ell'$ with $\ell \in L$ and $\ell' \in L'$.

\begin{lem}\label{1.12}
	An element $f$ of $F$ is an algebraic integer if and only if there exists a non-zero lattice $L \subset F$ with $fL \subset L$.
\end{lem}

\begin{proof}
	For if $f$ is a root of the polynomial $x^{k}+a_{1} x^{k-1}+\cdots+a_{k}$ with coefficients in $\Z$, then the field elements $1, f, f^{2}, \ldots f^{k-1}$ span a lattice $L=\Z[f]$ with $fL \subset L$. Conversely, if $fL \subset L$ where $L$ is spanned by $b_{1}, \ldots, b_{k}$, then we can set
	\[
		fb_{i}=\sum_{j} a_{ij} b_{j}
	\]
	for some matrix $\left(a_{i j}\right)$ of rational integers. Writing this as
	\[
		\sum_{j}\left(f \delta_{i j}-a_{i j}\right) b_{j}=0,
	\]
	where $\left(\delta_{ij}\right)$ denotes the $k \times k$ identity matrix, it follows that the columns of the matrix ($f \delta_{i j}-a_{i j}$) are linearly dependent. Therefore $f$ satisfies the monic polynomial equation
	\[
		\det\left(f \delta_{i j}-a_{i j}\right)=0
	\]
	with coefficients in $\Z$; which proves \fullref{1.12}.
\end{proof}

It follows that the set $\Lambda$, consisting of all algebraic integers in F , is closed under addition and multiplication. For if $\lambda, \mu \in \Lambda$, then there exist lattices $L$ and $L'$ with
\[
\lambda L \subset L, \quad \mu L' \subset L'
\]
Now the product lattice $L''=L'$ will satisfy $(\lambda+\mu) L'' \subset L''$ and $\lambda \mu L'' \subset L''$. Thus $\Lambda$ is a ring.

\begin{lem}\label{1.13}
	This set $\Lambda \subset F$ is itself a lattice of rank $n$ in $F$.
\end{lem}

\begin{proof}
The proof will be based on the trace homomorphism from $F$ to $\Q$. If $F' \supset F$ is a Galois extension of $\Q$ , recall that the additive homomorphism
\[
\tr_{F / \Q}: F \to \Q
\]
can be defined by the formula
\[
\tr_{F / \Q}(f)=\sigma_{1}(f)+\cdots+\sigma_{n}(f),
\]
where $\sigma_{1}, \ldots, \sigma_{n}$ are the distinct embeddings of $F$ in $F'$. (See, for example, \cite{Lang2012})

Note that the set of algebraic integers in $\Q$ is precisely equal to $\Z$. For if a fraction $\frac{a}{b} $ satisfies a monic polynomial equation with coefficients in $\Z$, then clearing denominators we see that every prime which divides $b$ must also divide $a$.

Therefore the trace homomorphism from $F$ to $\Q$ maps $\Lambda$ into $\Z$. For if $\lambda \in \Lambda$, then trace $F / \Q(\lambda)$ is both an algebraic integer (in $F'$) and a rational number; hence $\tr_{F / \Q}(\lambda) \in \Z$.

Note that every field element $f$ possesses a multiple $f+f+\cdots+f=mf$ which is an algebraic integer. In fact, expressing $f$ as the root of a polynomial with integer coefficients, we can take $m$ to be the absolute value of the leading coefficient. It follows that the quotient field of $\Lambda$ is equal to $F$.

Consider the $\Q$-bilinear pairing
\[
f, f' \mapsto \tr_{F / \Q}\left(ff'\right)
\]
from $F \times F$ to $\Q$. This pairing is non-degenerate, since for each $f \neq 0$ we can choose $f'=1 / f$ so that $\tr(ff') \neq 0$. Choose algebraic integers $\lambda_{1}, \ldots, \lambda_{n}$ which form a basis for $F$ over $\Q$ . Then the $\Q$-linear function
\[
f \mapsto\left(\tr_{F / \Q}\left(\lambda_{1} f\right), \ldots, \tr_{F / \Q}\left(\lambda_{n} f\right)\right)
\]
from $F$ to $\Q \oplus \cdots \oplus \Q$ is bijective, and embeds $\Lambda$ in the direct sum $\Z \oplus \cdots \oplus \Z$. Therefore $\Lambda$ is finitely generated as additive group; which proves \fullref{1.13}.
\end{proof}

It follows that every non-zero ideal $\mathfrak{a} \subset \Lambda$ is also a lattice of rank $n$ in $F$. Here are three important consequences of \fullref{1.13}.

\begin{enumerate}
	\item \textit{The ring $\Lambda$ is Noetherian.}\\
	In fact, if $\mathfrak{a}$ is a non-zero ideal, then $\Lambda / \mathfrak{a}$ is finite, so there are only finitely many larger ideals.
\item \textit{Every non-zero prime ideal of $\Lambda$ is maximal.}\\
	For the quotient ring $\Lambda / \mathfrak{p}$, being finite and without zero divisors, must be a field.
\item \textit{If an element $f$ in the quotient field of $\Lambda$ satisfies $f \mathfrak{a} \subset \mathfrak{a}$ for some non-zero ideal $\mathfrak{a}$, then $f \in \Lambda$.}\\
	In other words $\Lambda$ is integrally closed in its quotient field. This follows using \fullref{1.12}.
\end{enumerate}

We will show that any domain satisfying (1), (2) and (3) is necessarily Dedekind. The proof, due to van der Waerden, is based on the following.

\begin{obs}
	Every non-zero ideal a in a commutative Noetherian ring contains a product of non-zero prime ideals.
\end{obs}

For if $\mathfrak{a}$ itself is prime, there is nothing to prove. Otherwise, choosing ring elements $\lambda$ and $\mu$ not in $\mathfrak{a}$ so that $\lambda \mu \in \mathfrak{a}$, the two ideals $\mathfrak{a}+\lambda \Lambda$ and $\mathfrak{a}+\mu \Lambda$ are strictly larger than $\mathfrak{a}$, but have product contained in $\mathfrak{a}$. Assuming inductively that the Observation is true for these two larger ideals, it follows that it is true for $\mathfrak{a}$ also. (This ``induction'' argument makes sense since $\Lambda$ is Noetherian.)

Now given a domain $\Lambda$ satisfying (1), (2) and (3), and given non-zero ideals $\mathfrak{a} \subset \mathfrak{b}$, we must show that $\mathfrak{a}=\mathfrak{b} \mathfrak{c}$ for some ideal $\mathfrak{c}$. We will assume inductively that this statement is true for any ideal $\mathfrak{b}'$ which is strictly larger than $\mathfrak{b}$; and for any $\mathfrak{a}' \subset \mathfrak{b}'$. To start the induction, the statement is certainly true when $\mathfrak{b}=\Lambda$.

Choose an element $b \neq 0$ in $\mathfrak{b}$, and choose a product $\mathfrak{p}_{1} \cdots \mathfrak{p}_{r}$ of maximal ideals so that $\mathfrak{p}_{1} \cdots \mathfrak{p}_{r} \subset \Lambda b$, with $r$ minimal. Also choose a maximal ideal $\mathfrak{p} \supset \mathfrak{b}$. Then $\mathfrak{p}$ contains the product $\mathfrak{p}_{1} \cdots \mathfrak{p}_{\mathfrak{r}}$, hence $\mathfrak{p}$ contains some $\mathfrak{p}_{i}$, and therefore $\mathfrak{p}=\mathfrak{p}_{i}$. To fix our ideas, assume that $\mathfrak{p}=\mathfrak{p}_{1}$. The product $\mathfrak{p}_{2} \cdots \mathfrak{p}_{r}$ is not contained in $\Lambda b$, since $r$ is minimal, so there exists an element $c \in \mathfrak{p}_{2} \cdots \mathfrak{p}_{r}$ with $c \notin \Lambda b$. Evidently
\[
	c\mathfrak{b} \subset c\mathfrak{p} \subset \mathfrak{p}_{2} \cdots \mathfrak{p}_{r} \mathfrak{p}=\mathfrak{p}_{1} \cdots \mathfrak{p}_{r} \subset \Lambda b
\]
Therefore
\[
(c / b) \mathfrak{b} \subset \Lambda
\]
even though the element $c / b \in F$ does not belong to $\Lambda$. Consider the ideal
\[
\mathfrak{b}'=b^{-1}(\Lambda b+\Lambda c) \mathfrak{b}=\mathfrak{b}+(c / b) \mathfrak{b}
\]
in $\Lambda$. Since $c / b \notin \Lambda$, it follows from (3) that $\mathfrak{b}'$ is strictly larger than $\mathfrak{b}$. Therefore, by the induction hypothesis, given any $\mathfrak{a} \subset \mathfrak{b}$ the equation
\[
\mathfrak{a}=\mathfrak{b}' \mathfrak{c}'
\]
has a solution c'. Setting
\[
	c=b^{-1}(\Lambda b+\Lambda c) \mathfrak{b}\mathfrak{c}' \subset F
\]
we have
\[
\mathfrak{b} \mathfrak{c}=b^{-1}(\Lambda \mathfrak{b}+\Lambda \mathfrak{c}) \mathfrak{b} \mathfrak{c}'=\mathfrak{b}' \mathfrak{c}'=\mathfrak{a},
\]
are required. This set $\mathfrak{c}$ is actually contained in $\Lambda$, since it satisfies the condition $\mathfrak{b} \mathfrak{c} \subset \mathfrak{b}$. (Compare (3).) This shows that $\Lambda$ is a Dedekind domain, and completes the proof of \fullref{thm:1.4}.
